\documentclass{beamer}
\usetheme{Frankfurt}

\usepackage{subfig}
\newsavebox{\measurebox}
\usepackage{pbox}
\usepackage[makeroom]{cancel}
\renewcommand{\CancelColor}{\color{red}}
\usepackage{verbatim}
\usepackage[english]{babel}
\usepackage[utf8x]{inputenc}
\usepackage{multicol}
\usepackage{hyperref}

\title{BCSG - Football, Data Scientist Application}
\author[BCSG - Football, Data Scientist Application]{Guillermo Vinu\'e Vis\'us \\ PhD in Statistics \\ Data scientist}
\institute[Guillermo Vinu\'e Vis\'us, PhD]{}
\vspace{0.075cm}
\date{\textcolor{brown}{{Valencia, Spain \\ October 2022}} \\
\vspace{0.2cm}
\hspace*{0.2cm}\includegraphics[width = 0.1\textwidth]{figures/bcsg.png}
}

\useoutertheme[subsection=false]{miniframes}
\usepackage{etoolbox}
\makeatletter
\patchcmd{\slideentry}{\advance\beamer@xpos by1\relax}{}{}{}
\def\beamer@subsectionentry#1#2#3#4#5{\advance\beamer@xpos by1\relax}%
\makeatother

\setbeamersize{text margin left = 4mm, text margin right = 4mm} 
\definecolor{wicblue}{RGB}{45,149,210}
\setbeamercolor{structure}{fg=wicblue}
\setbeamercolor{section in head/foot}{bg=wicblue}
\setbeamercolor{author in head/foot}{bg=wicblue}
\setbeamercolor{item}{fg=wicblue}

\setbeamertemplate{navigation symbols}{}
\makeatletter
\defbeamertemplate*{headline}{my smoothbars theme}
{%
  \pgfuseshading{beamer@barshade}%
  \ifbeamer@sb@subsection%
    \vskip-9.75ex%
  \else%
    \vskip-7ex%
  \fi%
  \begin{beamercolorbox}[ignorebg,ht=2.25ex,dp=3.75ex]{section in head/foot}
    \insertnavigation{.9\paperwidth}\hfill\insertframenumber/\inserttotalframenumber\hspace{.5em}
  \end{beamercolorbox}%
  \ifbeamer@sb@subsection%
    \begin{beamercolorbox}[ignorebg,ht=2.125ex,dp=1.125ex,%
      leftskip=.3cm,rightskip=.3cm plus1fil]{subsection in head/foot}
      \usebeamerfont{subsection in head/foot}\insertsubsectionhead
    \end{beamercolorbox}%
  \fi%
}%
\makeatother		
		
\setbeamertemplate{footline}{
\hbox{\begin{beamercolorbox}[wd=\paperwidth, ht = 2.5ex, dp = 1ex, leftskip = 0.3cm, rightskip = 0.3cm]{author in head/foot}
\usebeamerfont{author in head/foot}\insertshortauthor\hfill\insertshortinstitute\end{beamercolorbox}}}

\newlength\imagewidth
\newlength\imagescale

\begin{document}

\begin{frame}[plain]
\titlepage
\end{frame}

\section{General approach}
\begin{frame}{}
\begin{enumerate}
\scriptsize
\item Visualize where the players are located at the first timestamp with data available. By means of a visual inspection, we will be able to discern where each team attacked at the first period and consequently at the second one. \textcolor{red}{Tactical units are defined differently depending on what direction the players attack}.

\vspace*{0.25cm}

\item Once we set the attacking direction of each team, use the tracking $x$ coordinate from each timestamp and period to assign the tactical units for every player.

\vspace*{0.35cm}

\item Count the number of times that each tactical unit is assigned for each player and compute the corresponding percentages.
\end{enumerate}

\begin{figure}[H]
\centering
\begin{tabular}{cc}
\hspace*{-0.2cm}\includegraphics[width=0.48\textwidth]{figures/football_field_defensive.png}&
\includegraphics[width=0.48\textwidth]{figures/football_field_offensive.png}\\
\end{tabular}
\end{figure}
\end{frame}


\section{Results}
\begin{frame}{}

\begin{minipage}{0.3\textwidth} 
\begin{figure}[H]
\begin{center}
\vspace*{0.2cm}\includegraphics[width=1.7\textwidth]{figures/paris_marseille.png}
\end{center}
\end{figure} 
\end{minipage}
\   \
\hfill \begin{minipage}{6cm}
\vspace*{-0.1cm}
\begin{itemize}
\scriptsize
\item The plot allows us to infer where the team is positioned at the start of the match.
\item The table shows the assignation of a tactical unit and the percentages associated with the three of them for each player.
\end{itemize}
\end{minipage}

\begin{figure}[H]
\vspace*{-0.5cm}
\includegraphics[width=0.86\textwidth]{figures/marseille.png}
\end{figure}

\end{frame}

\section{Potential improvements}
\begin{frame}{}

\begin{itemize}
\footnotesize
\item The method may refine the assignation of tactical units if we knew what happened at every timestamp, that is to say, if we had events data. 

\vspace*{0.6cm}

\item We could for example know when defenders just went to the opposing area because of a corner kick. 

\vspace*{0.6cm}

\item That would allow us to remove these sporadic situations and focus on the usual players' range of action. 
\end{itemize}

\end{frame}

\section{Applications}
\begin{frame}{}

\begin{itemize}
\footnotesize
\item To find out if players form a solid block by joining defensive and offensive lines as close as possible.

\vspace*{0.6cm}

\item To identify players who are either more dynamic or more static and provide them with this feedback.

\vspace*{0.6cm}

\item To analyze the teams' style of play regarding how many of their players perform mostly in defense or in offense.

\vspace*{0.6cm}

\item To redefine players according their movement. For example, we can identify right/left defenders who join often the attack or strikers who help in defense.
\end{itemize}

\end{frame}

\section{Software}
\begin{frame}{}

\begin{itemize}
\footnotesize
\item Software tools used have been R (including R Markdown) and \LaTeX{}.

\vspace*{0.6cm}

\item Three R functions have been created. A general R script generates the final results.

\vspace*{0.6cm}

\item An html interactive report shows these results and discusses them.

\vspace*{0.6cm}

\item The link to access all the source files is \url{https://github.com/guivivi/bcsg}
\end{itemize}

\end{frame}

\end{document}
